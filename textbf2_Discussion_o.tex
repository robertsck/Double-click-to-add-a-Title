\textbf{2. Discussion of Issues}

Traditional security architectures and frameworks will not be
sufficient. Unique properties of IoT prevent their application.

Achieve security and privacy within an IoT environment is more
challenging than traditional computing environments due to the many
unique properties of the IoT:

• Pervasiveness - IoT devices are ubiquitous and many times invisible.
Sensors exist and are recording data that users are not even aware of.
The proliferation of sensors means a tremendous amount of data about
ourselves is being captured, mostly without explicit consent, and stored
by both government and private entities.

\begin{quote}
• "Unless social, legal, or technical forces intervene, it is
conceivable that there will be no place on earth where an ordinary
person will be able to avoid surveillance.\footnote{A Michael Froomkin,
  ``The Death of Privacy?,'' \emph{Stanford Law Review} 52, (2000 ay):
  1461-1543.}"
\end{quote}

• Heterogeneity - "IoT systems incorporate a wide variety of
interconnected devices that create interoperability challenges. IoT
interconnectivity naturally leads to interaction of systems and
components that are built by different vendors, according to different
standards, and using different protocols. The magnitude of the diversity
in IoT environments is extensive and introduces interoperability
challenges that can lead to substantial system vulnerability.\footnote{ACM,
  \emph{Public Comment on the Benefits, Challenges, and Potential Roles
  for the Government in Fostering the Advancement of the Internet of
  Things textendash Docket No. 160331306-6306-01} , ed. ACM:2016vx (ul
  2016), 5.}"

• Lack of security by design -

\begin{quote}
• "Although the folks who make dishwashers may be fantastic engineers,
or even great computer programmers, it doesn't necessarily imply they're
equipped to protect Internet users from the outset.\footnote{Brian Fung,
  ``Here's the Scariest Part About the Internet of Things,'' \emph{The
  Washington Post}, November 19, 2013.}"

• "Based on the results of their analysis, the Veracode team concluded
that the designers of the tested devices "weren't focused enough on
security and privacy, as a priority, putting consumers at risk for an
attack or physical intrusion."\footnote{Lucian Constantin,
  ``Researchers: Iot Devices Are Not Designed With Security in Mind,''
  \emph{InfoWorld},}"
\end{quote}

• Limited ability to update and patch -

\begin{quote}
• "The problem with this process is that no one entity has any
incentive, expertise, or even ability to patch the software once it's
shipped. The chip manufacturer is busy shipping the next version of the
chip, and the ODM is busy upgrading its product to work with this next
chip. Maintaining the older chips and products just isn't a priority."
"To make matters worse, it's often impossible to patch the software or
upgrade the components to the latest version." "Even when a patch is
possible, it's rarely applied." "The result is hundreds of millions of
devices that have been sitting on the Internet, unpatched and insecure,
for the last five to ten years.\footnote{BRUCE SCHNEIER, ``The Internet
  of Things is Wildly Insecure --- and Often Unpatchable,''
  \emph{Wired}, Jan 6, 2014.}"
\end{quote}

• Lack of user interface - Many devices have a minimal or no user
interface. This several limits the ability for a device to provide
Privacy Notice or obtaining user Consent.

\begin{quote}
• "A related concern is that many IoT devices -- such as home appliances
or medical devices -- have no screen or other interface to communicate
with the consumer, thereby making notice on the device itself difficult,
if not impossible. For those devices that do have screens, the screens
may be smaller than even the screens on mobile devices, where providing
notice is already a challenge. Finally, even if a device has screens,
IoT sensors may collect data at times when the consumer may not be able
to read a notice (for example, while driving).\footnote{FTC,
  \emph{Internet of Things}, ed. FTC:2015tk (2015 an), 22.}
\end{quote}

• High degree of data sharing

\begin{quote}
• "It is important to recognize that the value of IoT is in ecosystems
rather than the individual component or cross-device interactions. This
goes beyond the standard concept of interoperability and composability
at the communications or the software level and into information
semantics. For example, data within streams that flow between IoT
devices and sensors can mean different things to different
components.\footnote{ACM, \emph{Public Comment on the Benefits,
  Challenges, and Potential Roles for the Government in Fostering the
  Advancement of the Internet of Things textendash Docket No.
  160331306-6306-01}, 4.}

• In addition, the presence of data could be used in different ways by
different devices, which presents a challenge for Privacy notice and
choice.
\end{quote}

• Composability

\begin{quote}
• "Composability will be a technical issue to consider, particularly
given the large number of IoT devices and sensors that interact with
each other. As a unit, a device or sensor may meet security, privacy,
and safety requirements. However, when combined or integrated with other
devices and sensors, as expected in IoT, there is no certainty that
these properties will remain. In a composable infrastructure, systems
can assemble in a variety of combinations based on user needs. The
integration of all these properties and behaviors brings opportunity but
also can have unintended consequences for the IoT ecosystem.\footnote{Ibid.}
\end{quote}

• High degree of data volume

\begin{quote}
• "Refers to the massive amounts of data that IoT components capture
that directly relate to human activity. The large volume of sometimes
highly personal data can be used in unintended ways, like to create
detailed predictive profiles of individuals. Moreover, the availability
of IoT data creates new privacy risks when combined with existing data
sources such as web and social data that can increase their predictive
power by combining online behaviors and behaviors in the physical
environment.\footnote{Ibid., 6.}

• As the devices and sensors within the IoT ecosystem become
increasingly pervasive, they will contribute to the volume of data
available, the velocity at which data will be generated, and the variety
of devices and sensors capturing data. The massive collection of data
and the new type and amount of data will likely reveal new insights. The
disparate individual pieces of information when combined can reveal
sensitive patterns that were previously not readily identifiable; this
is known as mosaic theory. This raises privacy concerns because data
collection, storage, and sharing might expose users to unexpected
privacy risks. Furthermore, data that is collected for one purpose may
allow inference of other information in ways that users and developers
may not expect. As IoT devices and sensors become integrated into daily
life, these risks will increase. They will be further exacerbated as
algorithmic power progresses and predictive data capabilities continue
to grow. We encourage further discussion on the various privacy concerns
related to transparency, accuracy, metadata maintenance, user
notification, data access, data usage, data attribution, and data
sharing. (From 24)
\end{quote}