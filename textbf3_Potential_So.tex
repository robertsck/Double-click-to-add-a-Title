\section{Potential Solutions}

Recent comments to the NTIA\footnote{See eg. "The IoT ecosystem is
  complex, and regulations should not impose unnecessary burdens on
  developers, impede innovation or growth, or create tech mandates."
  Jake Ward, ``Public Comments of the Application Developers Alliance on
  the Benefits, Challenges, and Potential Roles for the Government in
  Fostering the Advancement of the Internet of ThingsDocket No:
  160331306-6306-01,'' (May 23, 2016).. And "government should allow
  private actors to determine the standards that will allow loT
  technologies to reach their potential, recognizing that that process
  may take some time." J. Scott Kohler, ``Nest Response to Ntia Rfc,''
  (June 2, 2016).} and FTC\footnote{FTC, \emph{Internet of Things}, vii.}
make clear that industry prefers a self-regulatory approach to resolving
security and privacy issues. Such an approach, they argue, will ensure
that the rapid pace of innovation in the industry is not impacted by
government regulation. Benefits of a self-regulatory approach include:

• technology-specific regulation can prevent innovation\footnote{Kohler,
  ``Nest Response to Ntia Rfc'', 7.("As one example, well-intended
  legislative proposals to mandate specific power source requirements
  for battery-powered smoke alarms effectively preclude connected
  alternatives from the market notwithstanding the substantial benefits
  of those devices.")}

• developers of technology are best positioned to determine
most-appropriate security and privacy controls

• there are self-regulatory models in-place today such as:

\begin{quote}
• Payment Card Industry Data Security Standard (PCI DSS)

• Groupe Speciale Mobile Association (GSM)
\end{quote}

There are already a broad range of U.S. laws and regulations that cover
IoT:

• FTC Act provides authority to pursue "unfair or deceptive" practices.
The FTC has used its authority against companies with weak security and
privacy practices\footnote{Federal Trade Commission, ``Privacy and Data
  Security Update:2015,'' (January 2016).}. A recent court ruling has
affirmed the FTC position of using the "unfairness" doctrine to pursue
companies for inadequate security\footnote{UNITED STATES COURT OF
  APPEALS FOR THE THIRD CIRCUIT, ``Federal Trade Commission V. Wyndham
  Worldwide Corporation, No. 14-3514,'' (August 24, 2015).}.

• Most states have data privacy laws requiring consumer notification in
the case of a data breach\footnote{Find and insert list of state by
  state laws}. However, it has been pointed out that much of the data
being collected by IoT would fall outside the scope of most of these
laws\footnote{Scott R Peppet, ``Regulating the Internet of Things: First
  Steps Toward Managing Discrimination, Privacy, Security and Consent,''
  \emph{Tex. L. Rev.} 93, (2014), 139.("Thus, in a small minority of
  states, health- or fitness-related sensor data---such as data produced
  by a Breathometer, Fitbit, Nike+ FuelBand, blood-glucose monitor,
  blood-pressure monitor, or other device---could arguably be protected
  by the state's data-breach notification law. In most, theft or breach
  of such data would not trigger public notification. Moreover,
  \emph{none} of these state statutes would be triggered by
  data-security breaches into datasets containing other types of sensor
  data discussed in Part I. Driving- related data, for example, would
  nowhere be covered; location, accelerometer, or other data from a
  smartphone would nowhere be covered; smart grid data or data streaming
  out of Internet of Things home appliances would nowhere be covered.
  Put most simply, current data-security-breach notification laws are
  ill prepared to alert the public of security problems on the Internet
  of Things.")}.

• The Wiretap Act - "restricts the recording of private communications
without consent"\footnote{18 U.S.C. § 2510 \emph{et} seq.}

• Computer Fraud and Abuse Act - "prevents access to a computing device
without authorization from the owner"\footnote{18 U.S.C. § 1030.}

• Children's Online Privacy Protection Act (COPA) - restrict the "
collection, use, and disclosure of personal information from children
through Internet-connected services and through online services that are
directed to children\footnote{15 U.S.C. § 6501 \emph{et seq.;} 16 C.F.R.
  § 312.}.

• Fair Credit Reporting Act (FCRA) - "imposes strict limitations on the
use or disclosure of information for purposes of determining an
individual's eligibility for credit, insurance, or employment, ensuring
that consumers have visibility into and control over the data that is
used to determine their eligibility for crucial benefits"\footnote{15
  U.S.C. § 1681 \emph{et seq.;} 12 C.F.R. § 1022.}

• Video Privacy Protection Act - "limits disclosure and storage of
information about individuals' video viewing habits.\footnote{18 U.S.C.
  § 2710 \emph{et seq.}}"

"No system of data protection anywhere in the world has produced more
legal settlements, judgments, consent decrees and, perhaps most
importantly, corporate compliance programmes that seek to protect and
ensure privacy than the United States.\footnote{TASHA MANORANJAN ALAN
  CHARLES RAUL, VIVEK MOHAN, ``The Privacy, Data Protection and
  Cybersecurity Law Review,'' (November 2014).}"

FTC has been asking for new legislation to provide General data-security
legislation and broad-based privacy legislation:

• Security - "strong, flexible, and technology-neutral federal
legislation to strengthen its existing data security enforcement tools
and to provide notification to consumers when there is a security
breach. General data security legislation should protect against
unauthorized access to both personal information and device
functionality itself."\footnote{FTC, \emph{Internet of Things}, vii.}

• Privacy - "Commission staff thus again recommends that Congress enact
broad- based (as opposed to IoT-specific) privacy legislation. Such
legislation should be flexible and technology-neutral, while also
providing clear rules of the road for companies about such issues as how
to provide choices to consumers about data collection and use
practices."\footnote{Ibid., viii.}

Industry specific regulations - HIPAA, PCI DSS, HITECH, FSMA, NIST

• These may apply to some devices if they are enforced by a covered
party through their due-diligence requirements. For example, a health
care provider may require a health device manufacturer to meet the
security requirements of HIPAA and HITECH \emph{if} they contractually
mandate those requirements when adopting use of the device.